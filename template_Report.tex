\documentclass[]{report}


% Title Page
\title{Literature surrounding Pervasive Computing}
\author{Garrett Jordan\\G00305145\\Research Methods}
\date{December 4th, 2015}



\begin{document}
\maketitle


% ABSTRACT
% 
%

\begin{abstract}
Lee and Meier define the aims of pervasive computing as to embed computer system into every day life and environments without demanding the users explicit attention\cite{1}. Connected devices are now ubiquitous with even the most mundane appliances in our surroundings having sensors and network access. It is the intent of this paper to review the literature surrounding the challenges that this ubiquity brings to UNFINISHED.
\end{abstract}

% Introduction section
% 
%

\section*{Introduction}
The major challenges, outside of storage methods of the data, occur in the areas of data modelling and data protection. With vast rams of data collected it is important that the information can be garnered from it and that privacy and security are maintained.\\

% Context section
% 
%

\section*{Context}
The importance of context is emphasised by Castelli, Mamei, Rosi \& Zambonelli by stressing that  pervasive computing services should:
\begin{quote}
exploit contextual information both to adapt to user needs and to enable autonomic behaviour\cite{2}
\end{quote} 

The W4 model outlined by Castelli et al used Who, Where, Why and When to model contextual information from gathered data. These four 'Ws' are recorded as a set of key value pairs can more semantic information can by gleaned from the information. They also outline ways to graph the connections between other sets of 'Ws' through learned meaning. As an example if you spend most evenings and weekends in a particular location ( a Where) that can be inferred to be home.
Illari et al\cite{4} maintain the emphasis on both context and Semantic meaning by arguing that:
\begin{quote}
``exploiting semantic techniques in mobility data management can bring valuable benefits to many domains characterized by the mobility of users and moving objects in general, such as traffic management, urban dynamics analysis, ambient assisted living, emergency management, m-health, etc.''\cite{4}
\end{quote}







% Linked Data section
% 
%

\section{Open Linked Data}

\begin{quote}
``The Linked Data paradigm
involves practices to publish, share, and connect data on the
Web, and offers a new way of data integration and interoperability.''\cite{3}
\end{quote}



% Security section
% 
%

\section{Security}
Security


% Conclusion section
% 
%


\section{Conclusion}


% END MAIN BODY
% 
%

\medskip

\begin{thebibliography}{99}
\bibitem{1} 
Deirdre Lee and René Meier. 
\textit{ A hybrid approach to context modelling in large-scale pervasive computing environments.}\\ 
On Proceedings of the Fourth International ICST Conference on COMmunication System softWAre and middlewaRE (COMSWARE '09). ACM, New York, NY, USA, , Article 14 , 12 pages. DOI=http://dx.doi.org/10.1145/1621890.1621909.

\bibitem{2} 
Gabriella Castelli, Marco Mamei, Alberto Rosi, and Franco Zambonelli. 2009. Extracting high-level information from location data: the W4 diary example. Mob. Netw. Appl. 14, 1 (February 2009), 107-119. DOI=http://dx.doi.org/10.1007/s11036-008-0104-y
 
\bibitem{3} 
Theodore Dalamagas, Nikos Bikakis, George Papastefanatos, Yannis Stavrakas, and Artemis G. Hatzigeorgiou. 2012. Publishing life science data as linked open data: the case study of miRBase. In Proceedings of the First International Workshop on Open Data (WOD '12). ACM, New York, NY, USA, 70-77. DOI=http://dx.doi.org/10.1145/2422604.2422615
 
\bibitem{4}
Sergio Ilarri, Dragan Stojanovic, and Cyril Ray. 2015. Semantic management of moving objects. Expert Syst. Appl. 42, 3 (February 2015), 1418-1435. DOI=http://dx.doi.org/10.1016/j.eswa.2014.08.057



\end{thebibliography}

\end{document}          
